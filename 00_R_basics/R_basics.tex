% Options for packages loaded elsewhere
\PassOptionsToPackage{unicode}{hyperref}
\PassOptionsToPackage{hyphens}{url}
\PassOptionsToPackage{dvipsnames,svgnames,x11names}{xcolor}
%
\documentclass[
  letterpaper,
  DIV=11,
  numbers=noendperiod]{scrartcl}

\usepackage{amsmath,amssymb}
\usepackage{iftex}
\ifPDFTeX
  \usepackage[T1]{fontenc}
  \usepackage[utf8]{inputenc}
  \usepackage{textcomp} % provide euro and other symbols
\else % if luatex or xetex
  \usepackage{unicode-math}
  \defaultfontfeatures{Scale=MatchLowercase}
  \defaultfontfeatures[\rmfamily]{Ligatures=TeX,Scale=1}
\fi
\usepackage{lmodern}
\ifPDFTeX\else  
    % xetex/luatex font selection
\fi
% Use upquote if available, for straight quotes in verbatim environments
\IfFileExists{upquote.sty}{\usepackage{upquote}}{}
\IfFileExists{microtype.sty}{% use microtype if available
  \usepackage[]{microtype}
  \UseMicrotypeSet[protrusion]{basicmath} % disable protrusion for tt fonts
}{}
\makeatletter
\@ifundefined{KOMAClassName}{% if non-KOMA class
  \IfFileExists{parskip.sty}{%
    \usepackage{parskip}
  }{% else
    \setlength{\parindent}{0pt}
    \setlength{\parskip}{6pt plus 2pt minus 1pt}}
}{% if KOMA class
  \KOMAoptions{parskip=half}}
\makeatother
\usepackage{xcolor}
\usepackage[margin=1in]{geometry}
\setlength{\emergencystretch}{3em} % prevent overfull lines
\setcounter{secnumdepth}{5}
% Make \paragraph and \subparagraph free-standing
\makeatletter
\ifx\paragraph\undefined\else
  \let\oldparagraph\paragraph
  \renewcommand{\paragraph}{
    \@ifstar
      \xxxParagraphStar
      \xxxParagraphNoStar
  }
  \newcommand{\xxxParagraphStar}[1]{\oldparagraph*{#1}\mbox{}}
  \newcommand{\xxxParagraphNoStar}[1]{\oldparagraph{#1}\mbox{}}
\fi
\ifx\subparagraph\undefined\else
  \let\oldsubparagraph\subparagraph
  \renewcommand{\subparagraph}{
    \@ifstar
      \xxxSubParagraphStar
      \xxxSubParagraphNoStar
  }
  \newcommand{\xxxSubParagraphStar}[1]{\oldsubparagraph*{#1}\mbox{}}
  \newcommand{\xxxSubParagraphNoStar}[1]{\oldsubparagraph{#1}\mbox{}}
\fi
\makeatother

\usepackage{color}
\usepackage{fancyvrb}
\newcommand{\VerbBar}{|}
\newcommand{\VERB}{\Verb[commandchars=\\\{\}]}
\DefineVerbatimEnvironment{Highlighting}{Verbatim}{commandchars=\\\{\}}
% Add ',fontsize=\small' for more characters per line
\usepackage{framed}
\definecolor{shadecolor}{RGB}{241,243,245}
\newenvironment{Shaded}{\begin{snugshade}}{\end{snugshade}}
\newcommand{\AlertTok}[1]{\textcolor[rgb]{0.68,0.00,0.00}{#1}}
\newcommand{\AnnotationTok}[1]{\textcolor[rgb]{0.37,0.37,0.37}{#1}}
\newcommand{\AttributeTok}[1]{\textcolor[rgb]{0.40,0.45,0.13}{#1}}
\newcommand{\BaseNTok}[1]{\textcolor[rgb]{0.68,0.00,0.00}{#1}}
\newcommand{\BuiltInTok}[1]{\textcolor[rgb]{0.00,0.23,0.31}{#1}}
\newcommand{\CharTok}[1]{\textcolor[rgb]{0.13,0.47,0.30}{#1}}
\newcommand{\CommentTok}[1]{\textcolor[rgb]{0.37,0.37,0.37}{#1}}
\newcommand{\CommentVarTok}[1]{\textcolor[rgb]{0.37,0.37,0.37}{\textit{#1}}}
\newcommand{\ConstantTok}[1]{\textcolor[rgb]{0.56,0.35,0.01}{#1}}
\newcommand{\ControlFlowTok}[1]{\textcolor[rgb]{0.00,0.23,0.31}{\textbf{#1}}}
\newcommand{\DataTypeTok}[1]{\textcolor[rgb]{0.68,0.00,0.00}{#1}}
\newcommand{\DecValTok}[1]{\textcolor[rgb]{0.68,0.00,0.00}{#1}}
\newcommand{\DocumentationTok}[1]{\textcolor[rgb]{0.37,0.37,0.37}{\textit{#1}}}
\newcommand{\ErrorTok}[1]{\textcolor[rgb]{0.68,0.00,0.00}{#1}}
\newcommand{\ExtensionTok}[1]{\textcolor[rgb]{0.00,0.23,0.31}{#1}}
\newcommand{\FloatTok}[1]{\textcolor[rgb]{0.68,0.00,0.00}{#1}}
\newcommand{\FunctionTok}[1]{\textcolor[rgb]{0.28,0.35,0.67}{#1}}
\newcommand{\ImportTok}[1]{\textcolor[rgb]{0.00,0.46,0.62}{#1}}
\newcommand{\InformationTok}[1]{\textcolor[rgb]{0.37,0.37,0.37}{#1}}
\newcommand{\KeywordTok}[1]{\textcolor[rgb]{0.00,0.23,0.31}{\textbf{#1}}}
\newcommand{\NormalTok}[1]{\textcolor[rgb]{0.00,0.23,0.31}{#1}}
\newcommand{\OperatorTok}[1]{\textcolor[rgb]{0.37,0.37,0.37}{#1}}
\newcommand{\OtherTok}[1]{\textcolor[rgb]{0.00,0.23,0.31}{#1}}
\newcommand{\PreprocessorTok}[1]{\textcolor[rgb]{0.68,0.00,0.00}{#1}}
\newcommand{\RegionMarkerTok}[1]{\textcolor[rgb]{0.00,0.23,0.31}{#1}}
\newcommand{\SpecialCharTok}[1]{\textcolor[rgb]{0.37,0.37,0.37}{#1}}
\newcommand{\SpecialStringTok}[1]{\textcolor[rgb]{0.13,0.47,0.30}{#1}}
\newcommand{\StringTok}[1]{\textcolor[rgb]{0.13,0.47,0.30}{#1}}
\newcommand{\VariableTok}[1]{\textcolor[rgb]{0.07,0.07,0.07}{#1}}
\newcommand{\VerbatimStringTok}[1]{\textcolor[rgb]{0.13,0.47,0.30}{#1}}
\newcommand{\WarningTok}[1]{\textcolor[rgb]{0.37,0.37,0.37}{\textit{#1}}}

\providecommand{\tightlist}{%
  \setlength{\itemsep}{0pt}\setlength{\parskip}{0pt}}\usepackage{longtable,booktabs,array}
\usepackage{calc} % for calculating minipage widths
% Correct order of tables after \paragraph or \subparagraph
\usepackage{etoolbox}
\makeatletter
\patchcmd\longtable{\par}{\if@noskipsec\mbox{}\fi\par}{}{}
\makeatother
% Allow footnotes in longtable head/foot
\IfFileExists{footnotehyper.sty}{\usepackage{footnotehyper}}{\usepackage{footnote}}
\makesavenoteenv{longtable}
\usepackage{graphicx}
\makeatletter
\newsavebox\pandoc@box
\newcommand*\pandocbounded[1]{% scales image to fit in text height/width
  \sbox\pandoc@box{#1}%
  \Gscale@div\@tempa{\textheight}{\dimexpr\ht\pandoc@box+\dp\pandoc@box\relax}%
  \Gscale@div\@tempb{\linewidth}{\wd\pandoc@box}%
  \ifdim\@tempb\p@<\@tempa\p@\let\@tempa\@tempb\fi% select the smaller of both
  \ifdim\@tempa\p@<\p@\scalebox{\@tempa}{\usebox\pandoc@box}%
  \else\usebox{\pandoc@box}%
  \fi%
}
% Set default figure placement to htbp
\def\fps@figure{htbp}
\makeatother

% UW-themed PDF header
\usepackage{xcolor}
\definecolor{uwpurple}{HTML}{4B2E83}
\definecolor{uwgold}{HTML}{B7A57A}
\usepackage{titlesec}
\titleformat{\section}{\color{uwpurple}\normalfont\Large\bfseries}{\thesection}{1em}{}
\titleformat{\subsection}{\color{uwpurple}\normalfont\large\bfseries}{\thesubsection}{1em}{}
\titleformat{\subsubsection}{\color{uwpurple}\normalfont\normalsize\bfseries}{\thesubsubsection}{1em}{}
\usepackage{hyperref}
\hypersetup{colorlinks=true, linkcolor=uwpurple, citecolor=uwpurple, urlcolor=uwpurple}
\KOMAoption{captions}{tableheading}
\makeatletter
\@ifpackageloaded{caption}{}{\usepackage{caption}}
\AtBeginDocument{%
\ifdefined\contentsname
  \renewcommand*\contentsname{Table of contents}
\else
  \newcommand\contentsname{Table of contents}
\fi
\ifdefined\listfigurename
  \renewcommand*\listfigurename{List of Figures}
\else
  \newcommand\listfigurename{List of Figures}
\fi
\ifdefined\listtablename
  \renewcommand*\listtablename{List of Tables}
\else
  \newcommand\listtablename{List of Tables}
\fi
\ifdefined\figurename
  \renewcommand*\figurename{Figure}
\else
  \newcommand\figurename{Figure}
\fi
\ifdefined\tablename
  \renewcommand*\tablename{Table}
\else
  \newcommand\tablename{Table}
\fi
}
\@ifpackageloaded{float}{}{\usepackage{float}}
\floatstyle{ruled}
\@ifundefined{c@chapter}{\newfloat{codelisting}{h}{lop}}{\newfloat{codelisting}{h}{lop}[chapter]}
\floatname{codelisting}{Listing}
\newcommand*\listoflistings{\listof{codelisting}{List of Listings}}
\makeatother
\makeatletter
\makeatother
\makeatletter
\@ifpackageloaded{caption}{}{\usepackage{caption}}
\@ifpackageloaded{subcaption}{}{\usepackage{subcaption}}
\makeatother

\usepackage{bookmark}

\IfFileExists{xurl.sty}{\usepackage{xurl}}{} % add URL line breaks if available
\urlstyle{same} % disable monospaced font for URLs
\hypersetup{
  pdftitle={Getting Started with R},
  pdfauthor={HEOR DS 510},
  colorlinks=true,
  linkcolor={blue},
  filecolor={Maroon},
  citecolor={Blue},
  urlcolor={Blue},
  pdfcreator={LaTeX via pandoc}}


\title{Getting Started with R}
\author{HEOR DS 510}
\date{2025-11-04}

\begin{document}
\maketitle

\renewcommand*\contentsname{Table of contents}
{
\hypersetup{linkcolor=}
\setcounter{tocdepth}{3}
\tableofcontents
}

Introduction

\begin{itemize}
\tightlist
\item
  This guide shows how to:

  \begin{itemize}
  \tightlist
  \item
    Download and install R and RStudio
  \item
    Understand how base R differs from RStudio
  \item
    Navigate the RStudio environment (the 4 panes)
  \item
    Format flat files (like CSV) for loading
  \item
    Load a dataset (flat file and pointers to other resources)
  \item
    Install and load libraries
  \item
    Create objects
  \item
    Conduct basic analyses
  \item
    Save datasets
  \item
    Save R Markdown and Quarto files
  \end{itemize}
\end{itemize}

\section{Download and Install R and
RStudio}\label{download-and-install-r-and-rstudio}

\begin{itemize}
\tightlist
\item
  R is the programming language. RStudio (by Posit) is a popular
  Integrated Development Environment (IDE) for R.
\item
  Install R first, then RStudio.
\end{itemize}

Steps

\begin{enumerate}
\def\labelenumi{\arabic{enumi})}
\tightlist
\item
  Download R (CRAN):
\end{enumerate}

\begin{itemize}
\tightlist
\item
  https://cran.r-project.org/
\item
  Click ``Download R for Windows'' or ``macOS'' or ``Linux''
\item
  Choose the latest release, then run the installer
\end{itemize}

\begin{enumerate}
\def\labelenumi{\arabic{enumi})}
\setcounter{enumi}{1}
\tightlist
\item
  Download RStudio Desktop (free):
\end{enumerate}

\begin{itemize}
\tightlist
\item
  https://posit.co/download/rstudio-desktop/
\item
  Download the installer for your OS and run it
\item
  Open RStudio after installing (R should already be installed)
\end{itemize}

Check versions

\begin{Shaded}
\begin{Highlighting}[]
\NormalTok{R.version.string}
\end{Highlighting}
\end{Shaded}

\begin{verbatim}
[1] "R version 4.4.3 (2025-02-28)"
\end{verbatim}

\begin{Shaded}
\begin{Highlighting}[]
\CommentTok{\# If running inside RStudio, this may show RStudio version:}
\ControlFlowTok{if}\NormalTok{ (}\FunctionTok{exists}\NormalTok{(}\StringTok{"RStudio.Version"}\NormalTok{)) \{}
  \FunctionTok{try}\NormalTok{(\{}
\NormalTok{    v }\OtherTok{\textless{}{-}} \FunctionTok{RStudio.Version}\NormalTok{()}
    \FunctionTok{paste}\NormalTok{(}\StringTok{"RStudio version:"}\NormalTok{, v}\SpecialCharTok{$}\NormalTok{version)}
\NormalTok{  \}, }\AttributeTok{silent =} \ConstantTok{TRUE}\NormalTok{)}
\NormalTok{\}}
\end{Highlighting}
\end{Shaded}

\section{How Base R Differs from
RStudio}\label{how-base-r-differs-from-rstudio}

\begin{itemize}
\tightlist
\item
  Base R

  \begin{itemize}
  \tightlist
  \item
    The language + interpreter/engine
  \item
    Comes with a console and basic GUI (varies by OS)
  \item
    You can run scripts and commands via the R console or terminal
  \end{itemize}
\item
  RStudio IDE (by Posit)

  \begin{itemize}
  \tightlist
  \item
    A graphical environment wrapping R
  \item
    Offers script editor, plotting viewer, environment browser, package
    manager, project management, and integrated help
  \item
    Makes development, reproducibility, and visualization easier, but
    does not replace R itself
  \end{itemize}
\end{itemize}

Example: Base R functionality (works the same in RStudio since RStudio
calls R under the hood)

\begin{Shaded}
\begin{Highlighting}[]
\CommentTok{\# Base R calculations}
\NormalTok{x }\OtherTok{\textless{}{-}} \FunctionTok{c}\NormalTok{(}\DecValTok{1}\NormalTok{, }\DecValTok{2}\NormalTok{, }\DecValTok{3}\NormalTok{, }\DecValTok{4}\NormalTok{, }\DecValTok{5}\NormalTok{)}
\FunctionTok{mean}\NormalTok{(x)}
\end{Highlighting}
\end{Shaded}

\begin{verbatim}
[1] 3
\end{verbatim}

\begin{Shaded}
\begin{Highlighting}[]
\FunctionTok{sd}\NormalTok{(x)}
\end{Highlighting}
\end{Shaded}

\begin{verbatim}
[1] 1.581139
\end{verbatim}

\begin{Shaded}
\begin{Highlighting}[]
\FunctionTok{sum}\NormalTok{(x}\SpecialCharTok{\^{}}\DecValTok{2}\NormalTok{)}
\end{Highlighting}
\end{Shaded}

\begin{verbatim}
[1] 55
\end{verbatim}

\section{Introduction to the RStudio Environment (The 4
Panes)}\label{introduction-to-the-rstudio-environment-the-4-panes}

\begin{itemize}
\tightlist
\item
  Source (top-left, by default)

  \begin{itemize}
  \tightlist
  \item
    Your editor for .R scripts, .Rmd, .qmd files
  \item
    Run code lines or chunks into the Console
  \end{itemize}
\item
  Console (bottom-left)

  \begin{itemize}
  \tightlist
  \item
    Where R commands execute
  \item
    Shows outputs, errors, warnings
  \end{itemize}
\item
  Environment/History (top-right)

  \begin{itemize}
  \tightlist
  \item
    Environment: objects currently in memory
  \item
    History: previously run commands
  \end{itemize}
\item
  Files/Plots/Packages/Help/Viewer (bottom-right)

  \begin{itemize}
  \tightlist
  \item
    Files: browse project files
  \item
    Plots: view figures
  \item
    Packages: installed packages and load/unload controls
  \item
    Help: documentation for functions and packages
  \item
    Viewer: renders HTML content (e.g., Quarto docs)
  \end{itemize}
\end{itemize}

\begin{figure}[H]

{\centering \pandocbounded{\includegraphics[keepaspectratio]{images/Rstudio2.png}}

}

\caption{RStudio window}

\end{figure}%

Quick demonstrations

\begin{Shaded}
\begin{Highlighting}[]
\CommentTok{\# Create a few objects (watch the Environment pane update)}
\NormalTok{nums }\OtherTok{\textless{}{-}} \FunctionTok{rnorm}\NormalTok{(}\DecValTok{10}\NormalTok{)}
\NormalTok{df }\OtherTok{\textless{}{-}} \FunctionTok{data.frame}\NormalTok{(}\AttributeTok{id =} \DecValTok{1}\SpecialCharTok{:}\DecValTok{5}\NormalTok{, }\AttributeTok{value =} \FunctionTok{c}\NormalTok{(}\DecValTok{10}\NormalTok{, }\DecValTok{20}\NormalTok{, }\DecValTok{15}\NormalTok{, }\DecValTok{30}\NormalTok{, }\ConstantTok{NA}\NormalTok{))}

\CommentTok{\# Use Help pane: open documentation}
\FunctionTok{help}\NormalTok{(}\StringTok{"mean"}\NormalTok{)   }\CommentTok{\# or ?mean}

\CommentTok{\# Show a basic plot (appears in Plots tab)}
\FunctionTok{plot}\NormalTok{(nums, }\AttributeTok{type =} \StringTok{"b"}\NormalTok{, }\AttributeTok{main =} \StringTok{"Demo plot"}\NormalTok{, }\AttributeTok{xlab =} \StringTok{"Index"}\NormalTok{, }\AttributeTok{ylab =} \StringTok{"Value"}\NormalTok{)}
\end{Highlighting}
\end{Shaded}

\pandocbounded{\includegraphics[keepaspectratio]{R_basics_files/figure-pdf/unnamed-chunk-3-1.pdf}}

\section{Formatting Flat Files for
Loading}\label{formatting-flat-files-for-loading}

Good practices for CSV/TSV flat files

\begin{itemize}
\tightlist
\item
  Use a header row with short, clear, alphanumeric column names (avoid
  spaces; use underscores if needed)
\item
  Use UTF-8 encoding
\item
  Use a consistent delimiter (comma for CSV, tab for TSV)
\item
  Represent missing values consistently (e.g., empty cell or NA; avoid
  mixed values like ``-'', ``N/A'', ``null'')
\item
  Use ISO 8601 for dates (YYYY-MM-DD) and include time zones if
  timestamps are present
\item
  Avoid embedded line breaks in cells; if present, ensure proper quoting
\item
  Keep one ``tidy'' table per file: each row is one observation, each
  column is one variable
\end{itemize}

Create and save a well-formatted CSV

\begin{Shaded}
\begin{Highlighting}[]
\CommentTok{\# Example tidy dataset}
\NormalTok{tidy\_example }\OtherTok{\textless{}{-}} \FunctionTok{data.frame}\NormalTok{(}
  \AttributeTok{subject\_id =} \DecValTok{1}\SpecialCharTok{:}\DecValTok{6}\NormalTok{,}
  \AttributeTok{group =} \FunctionTok{c}\NormalTok{(}\StringTok{"control"}\NormalTok{, }\StringTok{"control"}\NormalTok{, }\StringTok{"control"}\NormalTok{, }\StringTok{"treatment"}\NormalTok{, }\StringTok{"treatment"}\NormalTok{, }\StringTok{"treatment"}\NormalTok{),}
  \AttributeTok{age\_years =} \FunctionTok{c}\NormalTok{(}\DecValTok{34}\NormalTok{, }\DecValTok{45}\NormalTok{, }\DecValTok{51}\NormalTok{, }\DecValTok{29}\NormalTok{, }\DecValTok{40}\NormalTok{, }\ConstantTok{NA}\NormalTok{),}
  \AttributeTok{visit\_date =} \FunctionTok{as.Date}\NormalTok{(}\FunctionTok{c}\NormalTok{(}\StringTok{"2025{-}01{-}10"}\NormalTok{, }\StringTok{"2025{-}01{-}12"}\NormalTok{, }\StringTok{"2025{-}01{-}13"}\NormalTok{, }\StringTok{"2025{-}01{-}11"}\NormalTok{, }\StringTok{"2025{-}01{-}12"}\NormalTok{, }\StringTok{"2025{-}01{-}14"}\NormalTok{)),}
  \AttributeTok{score =} \FunctionTok{c}\NormalTok{(}\DecValTok{87}\NormalTok{, }\DecValTok{90}\NormalTok{, }\DecValTok{85}\NormalTok{, }\DecValTok{92}\NormalTok{, }\DecValTok{88}\NormalTok{, }\DecValTok{91}\NormalTok{)}
\NormalTok{)}

\CommentTok{\# Create a data folder, then save CSV}
\FunctionTok{dir.create}\NormalTok{(}\StringTok{"data"}\NormalTok{, }\AttributeTok{showWarnings =} \ConstantTok{FALSE}\NormalTok{)}
\NormalTok{csv\_path }\OtherTok{\textless{}{-}} \FunctionTok{file.path}\NormalTok{(}\StringTok{"data"}\NormalTok{, }\StringTok{"tidy\_example.csv"}\NormalTok{)}
\FunctionTok{write.csv}\NormalTok{(tidy\_example, csv\_path, }\AttributeTok{row.names =} \ConstantTok{FALSE}\NormalTok{, }\AttributeTok{na =} \StringTok{""}\NormalTok{)}
\NormalTok{csv\_path}
\end{Highlighting}
\end{Shaded}

\begin{verbatim}
[1] "data/tidy_example.csv"
\end{verbatim}

\section{Loading a Dataset (Flat File and Other
Resources)}\label{loading-a-dataset-flat-file-and-other-resources}

Load the CSV using base R

\begin{Shaded}
\begin{Highlighting}[]
\NormalTok{loaded\_base }\OtherTok{\textless{}{-}} \FunctionTok{read.csv}\NormalTok{(csv\_path, }\AttributeTok{stringsAsFactors =} \ConstantTok{FALSE}\NormalTok{)}
\FunctionTok{str}\NormalTok{(loaded\_base)}
\end{Highlighting}
\end{Shaded}

\begin{verbatim}
'data.frame':   6 obs. of  5 variables:
 $ subject_id: int  1 2 3 4 5 6
 $ group     : chr  "control" "control" "control" "treatment" ...
 $ age_years : int  34 45 51 29 40 NA
 $ visit_date: chr  "2025-01-10" "2025-01-12" "2025-01-13" "2025-01-11" ...
 $ score     : int  87 90 85 92 88 91
\end{verbatim}

\begin{Shaded}
\begin{Highlighting}[]
\FunctionTok{head}\NormalTok{(loaded\_base)}
\end{Highlighting}
\end{Shaded}

\begin{verbatim}
  subject_id     group age_years visit_date score
1          1   control        34 2025-01-10    87
2          2   control        45 2025-01-12    90
3          3   control        51 2025-01-13    85
4          4 treatment        29 2025-01-11    92
5          5 treatment        40 2025-01-12    88
6          6 treatment        NA 2025-01-14    91
\end{verbatim}

Load the CSV using readr (tidyverse) for better performance and type
control

\begin{Shaded}
\begin{Highlighting}[]
\CommentTok{\# Install readr if needed (run once; eval is FALSE so it won\textquotesingle{}t execute automatically)}
\CommentTok{\# install.packages("readr")}
\end{Highlighting}
\end{Shaded}

\begin{Shaded}
\begin{Highlighting}[]
\CommentTok{\# If readr is available, demonstrate its use safely}
\ControlFlowTok{if}\NormalTok{ (}\FunctionTok{requireNamespace}\NormalTok{(}\StringTok{"readr"}\NormalTok{, }\AttributeTok{quietly =} \ConstantTok{TRUE}\NormalTok{)) \{}
\NormalTok{  loaded\_readr }\OtherTok{\textless{}{-}}\NormalTok{ readr}\SpecialCharTok{::}\FunctionTok{read\_csv}\NormalTok{(csv\_path, }\AttributeTok{show\_col\_types =} \ConstantTok{FALSE}\NormalTok{)}
  \FunctionTok{print}\NormalTok{(loaded\_readr)}
\NormalTok{\}}
\end{Highlighting}
\end{Shaded}

\begin{verbatim}
# A tibble: 6 x 5
  subject_id group     age_years visit_date score
       <dbl> <chr>         <dbl> <date>     <dbl>
1          1 control          34 2025-01-10    87
2          2 control          45 2025-01-12    90
3          3 control          51 2025-01-13    85
4          4 treatment        29 2025-01-11    92
5          5 treatment        40 2025-01-12    88
6          6 treatment        NA 2025-01-14    91
\end{verbatim}

Handling column types and missing values explicitly with readr

\begin{Shaded}
\begin{Highlighting}[]
\ControlFlowTok{if}\NormalTok{ (}\FunctionTok{requireNamespace}\NormalTok{(}\StringTok{"readr"}\NormalTok{, }\AttributeTok{quietly =} \ConstantTok{TRUE}\NormalTok{)) \{}
\NormalTok{  loaded\_typed }\OtherTok{\textless{}{-}}\NormalTok{ readr}\SpecialCharTok{::}\FunctionTok{read\_csv}\NormalTok{(}
\NormalTok{    csv\_path,}
    \AttributeTok{col\_types =}\NormalTok{ readr}\SpecialCharTok{::}\FunctionTok{cols}\NormalTok{(}
      \AttributeTok{subject\_id =}\NormalTok{ readr}\SpecialCharTok{::}\FunctionTok{col\_integer}\NormalTok{(),}
      \AttributeTok{group =}\NormalTok{ readr}\SpecialCharTok{::}\FunctionTok{col\_factor}\NormalTok{(}\AttributeTok{levels =} \FunctionTok{c}\NormalTok{(}\StringTok{"control"}\NormalTok{, }\StringTok{"treatment"}\NormalTok{)),}
      \AttributeTok{age\_years =}\NormalTok{ readr}\SpecialCharTok{::}\FunctionTok{col\_double}\NormalTok{(),}
      \AttributeTok{visit\_date =}\NormalTok{ readr}\SpecialCharTok{::}\FunctionTok{col\_date}\NormalTok{(),}
      \AttributeTok{score =}\NormalTok{ readr}\SpecialCharTok{::}\FunctionTok{col\_double}\NormalTok{()}
\NormalTok{    ),}
    \AttributeTok{show\_col\_types =} \ConstantTok{FALSE}
\NormalTok{  )}
  \FunctionTok{str}\NormalTok{(loaded\_typed)}
\NormalTok{\}}
\end{Highlighting}
\end{Shaded}

\begin{verbatim}
spc_tbl_ [6 x 5] (S3: spec_tbl_df/tbl_df/tbl/data.frame)
 $ subject_id: int [1:6] 1 2 3 4 5 6
 $ group     : Factor w/ 2 levels "control","treatment": 1 1 1 2 2 2
 $ age_years : num [1:6] 34 45 51 29 40 NA
 $ visit_date: Date[1:6], format: "2025-01-10" "2025-01-12" ...
 $ score     : num [1:6] 87 90 85 92 88 91
 - attr(*, "spec")=
  .. cols(
  ..   subject_id = col_integer(),
  ..   group = col_factor(levels = c("control", "treatment"), ordered = FALSE, include_na = FALSE),
  ..   age_years = col_double(),
  ..   visit_date = col_date(format = ""),
  ..   score = col_double()
  .. )
 - attr(*, "problems")=<externalptr> 
\end{verbatim}

Reading Excel files

\begin{Shaded}
\begin{Highlighting}[]
\CommentTok{\# install.packages("readxl")  \# run once}
\ControlFlowTok{if}\NormalTok{ (}\FunctionTok{requireNamespace}\NormalTok{(}\StringTok{"readxl"}\NormalTok{, }\AttributeTok{quietly =} \ConstantTok{TRUE}\NormalTok{)) \{}
  \CommentTok{\# Example: readxl::read\_excel("data/example.xlsx", sheet = 1)}
  \CommentTok{\# (We won’t read here unless a file exists)}
\NormalTok{\}}
\end{Highlighting}
\end{Shaded}

\begin{verbatim}
NULL
\end{verbatim}

Other dataset resources

\begin{itemize}
\tightlist
\item
  Built-in datasets: datasets::mtcars, iris, airquality
\item
  Public datasets:

  \begin{itemize}
  \tightlist
  \item
    palmerpenguins: https://allisonhorst.github.io/palmerpenguins/
  \item
    TidyTuesday datasets: https://github.com/rfordatascience/tidytuesday
  \item
    UCI Machine Learning Repository: https://archive.ics.uci.edu/
  \item
    Kaggle: https://www.kaggle.com/datasets
  \item
    data.gov (US): https://data.gov/
  \item
    World Bank Data: https://data.worldbank.org/
  \end{itemize}
\end{itemize}

\section{Installing and Loading
Libraries}\label{installing-and-loading-libraries}

Install packages (run once; do not run inside production pipelines
without a lockfile)

\begin{Shaded}
\begin{Highlighting}[]
\CommentTok{\# Example install (set eval: false to avoid automatic install)}
\CommentTok{\# install.packages(c("tidyverse", "readr", "dplyr", "ggplot2", "readxl", "data.table"))}
\end{Highlighting}
\end{Shaded}

Load packages

\begin{Shaded}
\begin{Highlighting}[]
\CommentTok{\# Load if available; fall back gracefully if not}
\NormalTok{loaded\_pkgs }\OtherTok{\textless{}{-}} \FunctionTok{c}\NormalTok{()}
\ControlFlowTok{for}\NormalTok{ (pkg }\ControlFlowTok{in} \FunctionTok{c}\NormalTok{(}\StringTok{"dplyr"}\NormalTok{, }\StringTok{"ggplot2"}\NormalTok{)) \{}
  \ControlFlowTok{if}\NormalTok{ (}\FunctionTok{requireNamespace}\NormalTok{(pkg, }\AttributeTok{quietly =} \ConstantTok{TRUE}\NormalTok{)) \{}
    \FunctionTok{library}\NormalTok{(pkg, }\AttributeTok{character.only =} \ConstantTok{TRUE}\NormalTok{)}
\NormalTok{    loaded\_pkgs }\OtherTok{\textless{}{-}} \FunctionTok{c}\NormalTok{(loaded\_pkgs, pkg)}
\NormalTok{  \}}
\NormalTok{\}}
\NormalTok{loaded\_pkgs}
\end{Highlighting}
\end{Shaded}

\begin{verbatim}
[1] "dplyr"   "ggplot2"
\end{verbatim}

Notes

\begin{itemize}
\tightlist
\item
  Use install.packages(``packagename'') once per machine or project
\item
  Use library(packagename) in each session/script where needed
\item
  Consider project environments for reproducibility (e.g., renv)
\end{itemize}

\section{Creating Objects}\label{creating-objects}

Basic objects

\begin{Shaded}
\begin{Highlighting}[]
\CommentTok{\# Numeric and character vectors}
\NormalTok{a }\OtherTok{\textless{}{-}} \FunctionTok{c}\NormalTok{(}\DecValTok{10}\NormalTok{, }\DecValTok{20}\NormalTok{, }\DecValTok{30}\NormalTok{)}
\NormalTok{b }\OtherTok{\textless{}{-}} \FunctionTok{c}\NormalTok{(}\StringTok{"alpha"}\NormalTok{, }\StringTok{"beta"}\NormalTok{, }\StringTok{"gamma"}\NormalTok{)}

\CommentTok{\# Factors}
\NormalTok{grp }\OtherTok{\textless{}{-}} \FunctionTok{factor}\NormalTok{(}\FunctionTok{c}\NormalTok{(}\StringTok{"control"}\NormalTok{, }\StringTok{"treatment"}\NormalTok{, }\StringTok{"control"}\NormalTok{), }\AttributeTok{levels =} \FunctionTok{c}\NormalTok{(}\StringTok{"control"}\NormalTok{, }\StringTok{"treatment"}\NormalTok{))}

\CommentTok{\# Matrices}
\NormalTok{m }\OtherTok{\textless{}{-}} \FunctionTok{matrix}\NormalTok{(}\DecValTok{1}\SpecialCharTok{:}\DecValTok{9}\NormalTok{, }\AttributeTok{nrow =} \DecValTok{3}\NormalTok{)}

\CommentTok{\# Lists (heterogeneous containers)}
\NormalTok{my\_list }\OtherTok{\textless{}{-}} \FunctionTok{list}\NormalTok{(}\AttributeTok{nums =}\NormalTok{ a, }\AttributeTok{labels =}\NormalTok{ b, }\AttributeTok{flag =} \ConstantTok{TRUE}\NormalTok{)}

\CommentTok{\# Data frames (tabular)}
\NormalTok{df2 }\OtherTok{\textless{}{-}} \FunctionTok{data.frame}\NormalTok{(}\AttributeTok{id =} \DecValTok{1}\SpecialCharTok{:}\DecValTok{3}\NormalTok{, }\AttributeTok{group =}\NormalTok{ grp, }\AttributeTok{score =} \FunctionTok{c}\NormalTok{(}\DecValTok{88}\NormalTok{, }\DecValTok{92}\NormalTok{, }\DecValTok{85}\NormalTok{))}
\FunctionTok{str}\NormalTok{(df2)}
\end{Highlighting}
\end{Shaded}

\begin{verbatim}
'data.frame':   3 obs. of  3 variables:
 $ id   : int  1 2 3
 $ group: Factor w/ 2 levels "control","treatment": 1 2 1
 $ score: num  88 92 85
\end{verbatim}

\begin{Shaded}
\begin{Highlighting}[]
\CommentTok{\# Functions}
\NormalTok{add\_two }\OtherTok{\textless{}{-}} \ControlFlowTok{function}\NormalTok{(x) x }\SpecialCharTok{+} \DecValTok{2}
\FunctionTok{add\_two}\NormalTok{(}\DecValTok{5}\NormalTok{)}
\end{Highlighting}
\end{Shaded}

\begin{verbatim}
[1] 7
\end{verbatim}

\section{Conducting Analyses}\label{conducting-analyses}

Descriptive statistics (base R)

\begin{Shaded}
\begin{Highlighting}[]
\NormalTok{x }\OtherTok{\textless{}{-}} \FunctionTok{rnorm}\NormalTok{(}\DecValTok{100}\NormalTok{, }\AttributeTok{mean =} \DecValTok{50}\NormalTok{, }\AttributeTok{sd =} \DecValTok{10}\NormalTok{)}
\FunctionTok{summary}\NormalTok{(x)}
\end{Highlighting}
\end{Shaded}

\begin{verbatim}
   Min. 1st Qu.  Median    Mean 3rd Qu.    Max. 
  28.10   44.22   51.32   50.57   56.69   76.92 
\end{verbatim}

\begin{Shaded}
\begin{Highlighting}[]
\FunctionTok{mean}\NormalTok{(x); }\FunctionTok{median}\NormalTok{(x); }\FunctionTok{sd}\NormalTok{(x); }\FunctionTok{quantile}\NormalTok{(x, }\AttributeTok{probs =} \FunctionTok{c}\NormalTok{(}\FloatTok{0.25}\NormalTok{, }\FloatTok{0.5}\NormalTok{, }\FloatTok{0.75}\NormalTok{))}
\end{Highlighting}
\end{Shaded}

\begin{verbatim}
[1] 50.57004
\end{verbatim}

\begin{verbatim}
[1] 51.32155
\end{verbatim}

\begin{verbatim}
[1] 10.20233
\end{verbatim}

\begin{verbatim}
     25%      50%      75% 
44.21789 51.32155 56.68620 
\end{verbatim}

Group-wise summaries (dplyr, if available)

\begin{Shaded}
\begin{Highlighting}[]
\ControlFlowTok{if}\NormalTok{ (}\FunctionTok{requireNamespace}\NormalTok{(}\StringTok{"dplyr"}\NormalTok{, }\AttributeTok{quietly =} \ConstantTok{TRUE}\NormalTok{)) \{}
  \FunctionTok{library}\NormalTok{(dplyr)}
\NormalTok{  loaded\_base }\SpecialCharTok{\%\textgreater{}\%}
    \FunctionTok{group\_by}\NormalTok{(group) }\SpecialCharTok{\%\textgreater{}\%}
    \FunctionTok{summarise}\NormalTok{(}
      \AttributeTok{n =} \FunctionTok{n}\NormalTok{(),}
      \AttributeTok{mean\_score =} \FunctionTok{mean}\NormalTok{(score, }\AttributeTok{na.rm =} \ConstantTok{TRUE}\NormalTok{),}
      \AttributeTok{mean\_age =} \FunctionTok{mean}\NormalTok{(age\_years, }\AttributeTok{na.rm =} \ConstantTok{TRUE}\NormalTok{)}
\NormalTok{    )}
\NormalTok{\}}
\end{Highlighting}
\end{Shaded}

\begin{verbatim}
# A tibble: 2 x 4
  group         n mean_score mean_age
  <chr>     <int>      <dbl>    <dbl>
1 control       3       87.3     43.3
2 treatment     3       90.3     34.5
\end{verbatim}

Visualization (ggplot2, if available)

\begin{Shaded}
\begin{Highlighting}[]
\ControlFlowTok{if}\NormalTok{ (}\FunctionTok{requireNamespace}\NormalTok{(}\StringTok{"ggplot2"}\NormalTok{, }\AttributeTok{quietly =} \ConstantTok{TRUE}\NormalTok{)) \{}
  \FunctionTok{library}\NormalTok{(ggplot2)}
  \FunctionTok{ggplot}\NormalTok{(loaded\_base, }\FunctionTok{aes}\NormalTok{(}\AttributeTok{x =}\NormalTok{ group, }\AttributeTok{y =}\NormalTok{ score, }\AttributeTok{fill =}\NormalTok{ group)) }\SpecialCharTok{+}
    \FunctionTok{geom\_boxplot}\NormalTok{() }\SpecialCharTok{+}
    \FunctionTok{geom\_jitter}\NormalTok{(}\AttributeTok{width =} \FloatTok{0.1}\NormalTok{, }\AttributeTok{alpha =} \FloatTok{0.6}\NormalTok{) }\SpecialCharTok{+}
    \FunctionTok{labs}\NormalTok{(}\AttributeTok{title =} \StringTok{"Scores by Group"}\NormalTok{, }\AttributeTok{x =} \StringTok{"Group"}\NormalTok{, }\AttributeTok{y =} \StringTok{"Score"}\NormalTok{) }\SpecialCharTok{+}
    \FunctionTok{theme\_minimal}\NormalTok{()}
\NormalTok{\}}
\end{Highlighting}
\end{Shaded}

\pandocbounded{\includegraphics[keepaspectratio]{R_basics_files/figure-pdf/unnamed-chunk-15-1.pdf}}

Linear regression

\begin{Shaded}
\begin{Highlighting}[]
\CommentTok{\# Fit a simple model on built{-}in mtcars dataset}
\NormalTok{fit }\OtherTok{\textless{}{-}} \FunctionTok{lm}\NormalTok{(mpg }\SpecialCharTok{\textasciitilde{}}\NormalTok{ wt }\SpecialCharTok{+}\NormalTok{ cyl, }\AttributeTok{data =}\NormalTok{ mtcars)}
\FunctionTok{summary}\NormalTok{(fit)}
\end{Highlighting}
\end{Shaded}

\begin{verbatim}

Call:
lm(formula = mpg ~ wt + cyl, data = mtcars)

Residuals:
    Min      1Q  Median      3Q     Max 
-4.2893 -1.5512 -0.4684  1.5743  6.1004 

Coefficients:
            Estimate Std. Error t value Pr(>|t|)    
(Intercept)  39.6863     1.7150  23.141  < 2e-16 ***
wt           -3.1910     0.7569  -4.216 0.000222 ***
cyl          -1.5078     0.4147  -3.636 0.001064 ** 
---
Signif. codes:  0 '***' 0.001 '**' 0.01 '*' 0.05 '.' 0.1 ' ' 1

Residual standard error: 2.568 on 29 degrees of freedom
Multiple R-squared:  0.8302,    Adjusted R-squared:  0.8185 
F-statistic: 70.91 on 2 and 29 DF,  p-value: 6.809e-12
\end{verbatim}

T-test (group comparison)

\begin{Shaded}
\begin{Highlighting}[]
\CommentTok{\# Compare mpg for automatic vs manual transmissions}
\FunctionTok{t.test}\NormalTok{(mpg }\SpecialCharTok{\textasciitilde{}}\NormalTok{ am, }\AttributeTok{data =}\NormalTok{ mtcars)}
\end{Highlighting}
\end{Shaded}

\begin{verbatim}

    Welch Two Sample t-test

data:  mpg by am
t = -3.7671, df = 18.332, p-value = 0.001374
alternative hypothesis: true difference in means between group 0 and group 1 is not equal to 0
95 percent confidence interval:
 -11.280194  -3.209684
sample estimates:
mean in group 0 mean in group 1 
       17.14737        24.39231 
\end{verbatim}

Contingency table and chi-squared test

\begin{Shaded}
\begin{Highlighting}[]
\NormalTok{tbl }\OtherTok{\textless{}{-}} \FunctionTok{table}\NormalTok{(mtcars}\SpecialCharTok{$}\NormalTok{cyl, mtcars}\SpecialCharTok{$}\NormalTok{gear)}
\NormalTok{tbl}
\end{Highlighting}
\end{Shaded}

\begin{verbatim}
   
     3  4  5
  4  1  8  2
  6  2  4  1
  8 12  0  2
\end{verbatim}

\begin{Shaded}
\begin{Highlighting}[]
\FunctionTok{chisq.test}\NormalTok{(tbl)}
\end{Highlighting}
\end{Shaded}

\begin{verbatim}

    Pearson's Chi-squared test

data:  tbl
X-squared = 18.036, df = 4, p-value = 0.001214
\end{verbatim}

\section{Saving Datasets}\label{saving-datasets}

Save to CSV (portable)

\begin{Shaded}
\begin{Highlighting}[]
\CommentTok{\# Save mtcars as CSV}
\NormalTok{out\_csv }\OtherTok{\textless{}{-}} \FunctionTok{file.path}\NormalTok{(}\StringTok{"data"}\NormalTok{, }\StringTok{"mtcars\_export.csv"}\NormalTok{)}
\FunctionTok{dir.create}\NormalTok{(}\StringTok{"data"}\NormalTok{, }\AttributeTok{showWarnings =} \ConstantTok{FALSE}\NormalTok{)}
\FunctionTok{write.csv}\NormalTok{(mtcars, out\_csv, }\AttributeTok{row.names =} \ConstantTok{FALSE}\NormalTok{)}
\NormalTok{out\_csv}
\end{Highlighting}
\end{Shaded}

\begin{verbatim}
[1] "data/mtcars_export.csv"
\end{verbatim}

Save to RDS (preserves R types precisely, single object)

\begin{Shaded}
\begin{Highlighting}[]
\NormalTok{out\_rds }\OtherTok{\textless{}{-}} \FunctionTok{file.path}\NormalTok{(}\StringTok{"data"}\NormalTok{, }\StringTok{"mtcars.rds"}\NormalTok{)}
\FunctionTok{saveRDS}\NormalTok{(mtcars, out\_rds)}
\CommentTok{\# Load back}
\NormalTok{mtcars\_loaded }\OtherTok{\textless{}{-}} \FunctionTok{readRDS}\NormalTok{(out\_rds)}
\FunctionTok{identical}\NormalTok{(mtcars, mtcars\_loaded)}
\end{Highlighting}
\end{Shaded}

\begin{verbatim}
[1] TRUE
\end{verbatim}

Save multiple objects to .RData (workspace-like)

\begin{Shaded}
\begin{Highlighting}[]
\NormalTok{out\_rdata }\OtherTok{\textless{}{-}} \FunctionTok{file.path}\NormalTok{(}\StringTok{"data"}\NormalTok{, }\StringTok{"analysis\_objects.RData"}\NormalTok{)}
\NormalTok{obj1 }\OtherTok{\textless{}{-}} \DecValTok{123}
\NormalTok{obj2 }\OtherTok{\textless{}{-}} \FunctionTok{data.frame}\NormalTok{(}\AttributeTok{x =} \DecValTok{1}\SpecialCharTok{:}\DecValTok{3}\NormalTok{, }\AttributeTok{y =} \FunctionTok{c}\NormalTok{(}\StringTok{"a"}\NormalTok{, }\StringTok{"b"}\NormalTok{, }\StringTok{"c"}\NormalTok{))}
\FunctionTok{save}\NormalTok{(obj1, obj2, }\AttributeTok{file =}\NormalTok{ out\_rdata)}
\CommentTok{\# Load back}
\FunctionTok{rm}\NormalTok{(obj1, obj2)}
\FunctionTok{load}\NormalTok{(out\_rdata)}
\NormalTok{obj1; obj2}
\end{Highlighting}
\end{Shaded}

\begin{verbatim}
[1] 123
\end{verbatim}

\begin{verbatim}
  x y
1 1 a
2 2 b
3 3 c
\end{verbatim}

\section{Saving R Markdown and Quarto
Files}\label{saving-r-markdown-and-quarto-files}

Saving files

\begin{itemize}
\tightlist
\item
  In RStudio:

  \begin{itemize}
  \tightlist
  \item
    File -\textgreater{} New File -\textgreater{} Quarto Document
  \item
    File -\textgreater{} Save (choose .qmd extension)
  \item
    For R Markdown: File -\textgreater{} New File -\textgreater{} R
    Markdown, then Save as .Rmd
  \end{itemize}
\end{itemize}

Rendering (turn .qmd or .Rmd into HTML/PDF/Word)

\begin{Shaded}
\begin{Highlighting}[]
\CommentTok{\# Quarto render (requires Quarto installed)}
\CommentTok{\# install.packages("quarto") is not needed; Quarto is a separate tool you install from https://quarto.org/}
\CommentTok{\# From R you can call:}
\ControlFlowTok{if}\NormalTok{ (}\FunctionTok{requireNamespace}\NormalTok{(}\StringTok{"quarto"}\NormalTok{, }\AttributeTok{quietly =} \ConstantTok{TRUE}\NormalTok{)) \{}
  \CommentTok{\# quarto::quarto\_render("your\_document.qmd")}
\NormalTok{\}}
\end{Highlighting}
\end{Shaded}

\begin{verbatim}
NULL
\end{verbatim}

\begin{Shaded}
\begin{Highlighting}[]
\CommentTok{\# R Markdown render (requires rmarkdown package)}
\CommentTok{\# install.packages("rmarkdown")  \# run once}
\ControlFlowTok{if}\NormalTok{ (}\FunctionTok{requireNamespace}\NormalTok{(}\StringTok{"rmarkdown"}\NormalTok{, }\AttributeTok{quietly =} \ConstantTok{TRUE}\NormalTok{)) \{}
  \CommentTok{\# rmarkdown::render("your\_document.Rmd", output\_format = "html\_document")}
\NormalTok{\}}
\end{Highlighting}
\end{Shaded}

\begin{verbatim}
NULL
\end{verbatim}

Command-line rendering (outside R)

\begin{itemize}
\tightlist
\item
  Quarto:

  \begin{itemize}
  \tightlist
  \item
    Install Quarto: https://quarto.org/docs/get-started/
  \item
    In a terminal: quarto render your\_document.qmd
  \end{itemize}
\item
  R Markdown (legacy):

  \begin{itemize}
  \tightlist
  \item
    In RStudio: click ``Knit''
  \item
    Or in R: rmarkdown::render(``your\_document.Rmd'')
  \end{itemize}
\end{itemize}

Export formats

\begin{itemize}
\tightlist
\item
  HTML (default), PDF (requires LaTeX), Word (docx)
\item
  Choose output format in the YAML header or via render arguments
\end{itemize}

\section{Sources and Further Reading}\label{sources-and-further-reading}

\begin{itemize}
\tightlist
\item
  R (CRAN) downloads: https://cran.r-project.org/
\item
  RStudio Desktop by Posit: https://posit.co/download/rstudio-desktop/
\item
  Quarto documentation: https://quarto.org/docs/
\item
  R for Data Science (2e): https://r4ds.hadley.nz/
\item
  Advanced R (3e): https://adv-r.hadley.nz/
\item
  Tidyverse packages: https://www.tidyverse.org/
\item
  readr (fast reading/writing): https://readr.tidyverse.org/
\item
  dplyr (data manipulation): https://dplyr.tidyverse.org/
\item
  ggplot2 (visualization): https://ggplot2.tidyverse.org/
\item
  R Markdown (legacy): https://rmarkdown.rstudio.com/
\item
  Base R documentation (manuals):
  https://cran.r-project.org/manuals.html
\item
  RStudio IDE docs: https://docs.posit.co/ide/
\item
  Data import best practices (readr vignette):
  https://readr.tidyverse.org/articles/readr.html
\item
  Palmer Penguins dataset:
  https://allisonhorst.github.io/palmerpenguins/
\item
  TidyTuesday: https://github.com/rfordatascience/tidytuesday
\item
  UCI ML Repository: https://archive.ics.uci.edu/
\item
  Kaggle datasets: https://www.kaggle.com/datasets
\item
  data.gov: https://data.gov/
\end{itemize}




\end{document}
